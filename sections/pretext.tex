%%%
%%% Dados Pessoais
%%%
\autor{Álan Lívio Vasconcelos Guedes}
\autorR{Guedes, Álan Lívio Vasconcelos}

\orientador{Simone Diniz Junqueira Barbosa}{Prof.\textsuperscript{a}}
\orientadorR{Barbosa, Simone Diniz Junqueira}

%%%
%%% Título
%%%
\titulo{Estendendo linguagens multimídia para suportar interações multimodais}
\titulouk{Extending multimedia languages to support multimodal user
interactions}

%%%
%%% Data
%%%
\dia{29}
\mes{September}
\ano{2017}

%%%
%%% Dados do Programa
%%%
\cidade{Rio de Janeiro}
\CDD{004}
\departamento{Informática}
\programa{Informática}
\centro{Centro Técnico Científico}
\universidade{Pontifícia Universidade Católica do Rio de Janeiro}
\uni{PUC-Rio}

%%%
%%% Banca
%%%
\banca{%
	\membrodabanca{Sérgio Colcher}{Prof.}
	{Departamento de Informática}{PUC-Rio}
	\membrodabanca{Hugo Fuks}{Prof.}
	{Departamento de Informática}{PUC-Rio}
	\membrodabanca{Débora Christina Muchaluat Saade}{Prof.\textsuperscript{a}}
	{Instituto de Computação}{UFF}
	\membrodabanca{Carlos de Salles Soares Neto}{Prof.}
	{Departamento de Informática}{UFMA}
	\coordenador{Márcio da Silveira Carvalho}{Prof.}
}

%%%
%%% Currículo
%%%
\curriculo{%
	The author received his Bachelor (2009) and M.Sc.
	(2012) in Computer Science from the Federal University of Paraíba (UFPB),
	where
	he worked as researcher in Lavid Lab. Since 2013, the author acts as
	researcher in TeleMídia Lab at PUC-Rio. During his academic career, he
	participated in several research projects about TV systems from funding
	agencies, such as RNP and FINEP. In particularly, his researches contribute
	to Ginga and NCL specifications, which today are standards for DTV, IPTV and
	IBB.
}

%%%
%%% Dedicatória
%%%
\dedicatoria{%
	For Prof. Luiz Fernando Soares (\textit{in memoriam}).
}

%%%
%%% Agradecimentos
%%%
\agradecimentos{%
	First, I would like to thank my beloved family, especially my parents Rosa
	and Ednaldo, my brothers Alysson and Adrian, my grandparents Maria and Abel,
	my uncle Amauri~(\textit{in memoriam}), my aunts Socorro, Lúcia and Tânia, my
	cousins Emanuelle and Matheus, my sisters in law Rozalinda and Nara. They
	were always present in my life and taught me how the commitment to family and
	work make a person noble and dignified.

	I would deeply thank my advisor Luis Fernando~(\textit{in
	memoriam}) for given me the honor of working with him. I do not have
	words to truly express my immense gratitude and admiration. His excellent
	guidance inspired and shaped me into becoming a better researcher and person.

	I would like to thank my advisor Simone Barbosa for
	the guidance and patience. Moreover, I deeply appreciate the support she give
	me in a hard moment of my Phd and for suddenly accept me as an new student.

	I would like to thank everyone from TeleMídia Lab, especially Sergio
	Colcher, 	Alvaro da Veiga, Roberto Gerson, Guilherme Lima, Rodrigo Costa,
	Felipe 	Nagato, Rafael Diniz, Antonio Busson, Andre Brandão, Marcos Roriz,
	Francisco Sant'Anna, Márcio Moreno and others. I am proud to	be part of a
	research lab 	that focus not only on performs high level researches but also
	on training 	each of their member as researcher and as person. I shared lot
	moments, coffee and chocolates with them. In particularly, I deeply thank
	Roberto Gerson and Sergio Colcher for their support in my Phd, without them I
	would not be able to fulfill this work.

	I would also like to thank everyone from LAC lab, especially Marcos Roriz,
	Luis Talavera, Markus Endler, Felipe, Francisco, André MacDowell and Patrícia
	Carrion. I also shared lot moments, coffee and chocolates with them.

	I would like to thank lot of people that I meet during my PhD journey,
	especially my beloved girlfriend Lisseth Saavedra, my roommates Marcos Roriz,
	Eduardo	Araújo, Daniel Pires, Katia Vega, Derlyane, Thais Abreu, Ruberth
	Barros, the Brandilha couple André and Júlia, and Dalai Ribeiro, my friends
	in department of informatics, Vanessa Leite, Aline Saettler, André Moreira,
	Hugo Gualandi, Lívia Ruback, Paula Ceccon, Wallas and Patrícia Carrion, my
	friends in department of electrical engineering, Andy	Alvarez, Sandra,
	Keila, Carlos, Roxana, Mauricio, Jennifer, Marcelo, Teddy, Junior, Oscar,
	Elizabeth and Emersson. They made this journey with more happiness.

  I would like to thank all participants of my evaluation study.

	I would like to thank all professors and staff from department of
	informatics of PUC-Rio, especially Regina Zanon.


	Finally, I would like to thank CNPq, for their financial aid.
}

%%%
%%% Prechaves (Necessário!)
%%%
\catalogprekeywords{%
	\catalogprekey{Informática}%
}

%%%
%%% Chaves (keywords)
%%%
\chaves{%
	\chave{Linguagens Multimídia;}
	\chave{Interações Multimodais;}
	\chave{MUI;}
	\chave{Interações Multiusuário;}
	\chave{Nested Context Language;}
	\chave{NCL;}
	\chave{HTML}
}

\chavesuk{%
	\chave{Multimedia Languages;}
	\chave{Multimodal User Interactions;}
	\chave{MUI;}
	\chave{Multiuser User Interactions;}
	\chave{Nested Context Language;}
	\chave{NCL;}
	\chave{HTML}
	%
}

%%%
%%% Resumo (Português e Inglês)
%%%
\resumo{%
	Os recentes avanços em tecnologias de reconhecimento, como fala, toque e
	gesto, deram origem a uma nova classe de interfaces de usuário que não apenas
	explora múltiplas modalidades de interação, mas também permite múltiplos
	usuários interagindo. O desenvolvimento de aplicativos com interações
	multimodais e multiusuários trazem desafios para a sua especificação e
	execução. A especificação de uma aplicação multimodal é comumente o foco das
	pesquisas em interação multimodal, enquanto a especificação de sincronismos
	audiovisuais geralmente é o foco das pesquisas em multimídia. Nesta tese, com
	o objetivo de auxiliar a especificação de tais aplicações, buscamos integrar
	conceitos dessas duas pesquisas e propomos estender linguagens multimídia com
	entidades de primeira classe para suportar recursos multiusuário e
	multimodais. Essas entidades foram instanciadas nas linguagens NCL e HTML.
	Para avaliar nossa abordagem, realizamos uma avaliação com desenvolvedores NCL
	e HTML para capturar indícios de aceitação das entidades propostas e suas
	sintaxes nessas linguagens.
}

\resumouk{%
	Recent advances in recognition technologies, such as speech, touch and
	gesture, have given rise to a new class of user interfaces that does not only
	explore multiple modalities but also allows for multiple interacting users.
	The development of applications with both multimodal and multiuser
	interactions arise new specification and execution issues. The specification
	of multimodal application is commonly the focus of multimodal interaction
	research, while the specification of the synchronization of audiovisual media
	is usually the focus of multimedia research. In this thesis, aiming to assist
	the specification of such applications, we propose to integrate concepts from
	those two research areas and to extend multimedia languages with first-class
	entities to support multiuser and multimodal features. Those entities were
	instantiated in NCL and HTML. To evaluate our approach, we performed an
	evaluation with NCL and HTML developers to capture evidences of their
	acceptance of the proposed entities and instantiations in those languages.
}

%%%
%%% Epigrafe
%%%

%% \epigrafe{%
%%   no epigrafe}
%% \epigrafeautor{Wassily Kandinsky}
%% \epigrafelivro{Regards sur le passé}

%--------------------------------------------------------------- TEXT

%%%
%%% Misc.
%%% Caso a Tese utilize figuras ou gráficos coloridos colSocar true, caso
%%%contrário false.
%%%
\usecolour{true}

%%%
%%% Glossário
%%%
\tituloglossario{List of abreviations}
